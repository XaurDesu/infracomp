\documentclass[12pt,a4paper,teal]{bbe}
\usepackage{blindtext}
\usepackage{fancyvrb}
\begin{document}
	
	\chapter{Concurrency}
	Concurrency is the act of running different processes simultaneously 
	on a single machine by using separate processing instances, called Threads. 
	This practice is mainly done as a way to distribute the load 
	of a program on more than one thread. Most modern processing systems are made with
	concurrency and multithreading in mind and therefore, use it for managing different programs at
	the same time. The main reason we want to use concurrency in our systems is for:
	
	\begin{enumerate}
	
		\item Facilitating the modeling of different tasks
		\item Better problem encapsulement
		\item Better logic separation of problems
		\item Performance improvements at runtime
		\item A better suited architecture for modern systems 
	\end{enumerate}
	
	We can find concurrency at a bunch of different levels, from microcontrollers
	to programs that work in parallel in different systems for a common goal, depending
	on the infrastructure available, local processes are going to be the main interest
	of this notes, however, as distributed computing is not necessairly the focus of
	the course i'm making this notes for.

	There are different ways to expressing concurrency, such as:
	\paragraph{Incorporating specific instructions to a Language}

	\paragraph{Giving directives to a compiler.}

	\paragraph{using an API}

	\subsection{Concurrency Types.}
	Threads can be used in modern processors of any core count, even though 
	the simultaneous nature of it might seem to imply the requirement of a
	multicore systems, single-core machines might
	also use threads, equally, the number of threads can wildly surpass the number
	of cores for a system. such systems that have cores multitasking on different
	tasks use a pattern called multiprogramming. When this pattern is not possible,
	we call it a Batch System.
	
	\subsection{Sequential and Concurrent programming.}
	\begin{definition}
		Concurrent programming is generally defined by the use of multiple threads.

	\end{definition}

	\subsection{Data integrity on Concurrency}
	Data integrity can be tricky while working on concurrency.

	\subsection{Java-specific concurrency}
		\begin{remark}
			For the sake of this document, examples and code blocks will be in Java,
			but concurrency is supported in a bunch on different languages, such
			as Python, Swift (i'm helping them!) and C++.
		\end{remark}
	\subsubsection{Examples}
	\paragraph*{Declaration}
	\begin{verbatim}
	public class T extends Thread {
	    public void run() {
	    	//Thread action
	    }
	}
	\end{verbatim}
	

	\paragraph{Creation and Activation}
		\begin{verbatim}
			public t = new T(); //Creation
			t.start(); //Activation.	
		\end{verbatim}
	
	\paragraph{searching for a value in a Matrix}
	\begin{verbatim}
		public class T extends Thread {
		  private static int valor;
		  private static int tamano;
		  private static int [][] M;

		    public T (int i) {
			      id = i;
		    }
		    public void run() {
		    	int nElementos = M[id].length;
		        for(int j = 0; j < nElementos; j++) {
		            if(M[id][j] == valor) {
			            System.out.println(M[id][j]);
			        }
		        }
		    }
		}
	
	\end{verbatim}

	
	\subsection{Synchronization}
	In general, concurrency doesn't guarantee a specific execution plan, and could
	be potentially destructive given certain contexts. 
	\subsubsection{Signaling}
	\subsubsection{Mutual Exclusion}
	

\end{document}